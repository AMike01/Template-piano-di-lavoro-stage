%----------------------------------------------------------------------------------------
%	STAGE DESCRIPTION
%----------------------------------------------------------------------------------------
\section*{Scopo dello stage}
% Personalizzare inserendo lo scopo dello stage, cioè una breve descrizione
Lo scopo dello stage è quello di permettere allo studente di entrare in contatto con un’organizzazione che si occupa di sviluppo software.\\

Lo studente sarà chiamato a realizzare un applicativo web che permetta, tramite l’uso di servizi di intelligenza artificiale generativa messi a disposizione dai cloud provider, di generare degli allegati tecnici relativi alla descrizione di funzionalità di sviluppo software e disegno di infrastrutture cloud. \\

A livello operativo quindi la studente dovrà realizzare:
\begin{itemize}
    \item Un frontend dove caricare un template di documento word dove sarà popolato il contenuto generato dall’intelligenza artificiale;
    \item Realizzare un prompt dove l’utente può inserire le indicazioni dei contenuti tecnici da realizzare. Il prompt potrà avere anche delle facilitazioni per la stesura di documenti al fine di poter rendere le proposte paragonabili nel tempo e fornire un formato uniforme nel tempo;
    \item Sviluppare le logiche di interrogazione dei servizi cloud per la generazione dei contenuti e integrazione del contenuto con il documento;
\end{itemize}


Lo sviluppo del servizio dovrà utilizzare AWS Bedrock ma lo sviluppo dovrà essere modulare per consentire in un secondo momento di sostituire Bedrock con altri sistemi di intelligenza generativa senza interrompere il processo pensato.



