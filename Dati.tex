%----------------------------------------------------------------------------------------
%   USEFUL COMMANDS
%----------------------------------------------------------------------------------------

\newcommand{\dipartimento}{Dipartimento di Matematica ``Tullio Levi-Civita''}

%----------------------------------------------------------------------------------------
% 	USER DATA
%----------------------------------------------------------------------------------------

% Data di approvazione del piano da parte del tutor interno; nel formato GG Mese AAAA
% compilare inserendo al posto di GG 2 cifre per il giorno, e al posto di 
% AAAA 4 cifre per l'anno
\newcommand{\dataApprovazione}{30 Agosto 2024}

% Dati dello Studente
\newcommand{\nomeStudente}{Alberto}
\newcommand{\cognomeStudente}{Michelazzo}
\newcommand{\matricolaStudente}{2010007}
\newcommand{\emailStudente}{alberto.michelazzo@studenti.unipd.it}
\newcommand{\telStudente}{3458723666}

% Dati del Tutor Aziendale
\newcommand{\nomeTutorAziendale}{Michele}
\newcommand{\cognomeTutorAziendale}{Massaro}
\newcommand{\emailTutorAziendale}{m.massaro@vargroup.it}
\newcommand{\telTutorAziendale}{}
\newcommand{\ruoloTutorAziendale}{Tech Leader Mobile development e Project Manager}

% Dati dell'Azienda
\newcommand{\ragioneSocAzienda}{UAN Company SRL}
\newcommand{\indirizzoAzienda}{Via Piovola 138 50053 Empoli (FI)}
\newcommand{\sitoAzienda}{}
\newcommand{\emailAzienda}{}
\newcommand{\partitaIVAAzienda}{P.IVA 04672910488}

% Dati del Tutor Interno (Docente)
\newcommand{\titoloTutorInterno}{Prof.}
\newcommand{\nomeTutorInterno}{Tullio}
\newcommand{\cognomeTutorInterno}{Vardanega}

\newcommand{\prospettoSettimanale}{
     % Personalizzare indicando in lista, i vari task settimana per settimana
     % sostituire a XX il totale ore della settimana
    \begin{itemize}
        \item \textbf{Sprint 1 (Durata 1 settimana/40 ore)}

        Lo scopo del primo sprint di lavoro per lo studente è quello di andare ad apprendere:
        \begin{itemize}
            \item Metodologia di lavoro agile;
            \item Scrittura delle user stories specifiche del progetto di stage da realizzare;
            \item Apprendere le tecnologie e linguaggi di programmazione che saranno utilizzati per la realizzazione del progetto.
        \end{itemize}

        \item \textbf{Sprint 2 (Durata 3 Settimane/120 ore)} 

        Creazione della web app che permetta la gestione dei flussi di caricamento del documento word, creazione del prompting e flussi di interazione con l’utente.

        \item \textbf{Sprint 3 (Durata 3 Settimane/120 ore)} 

        Sviluppo di processi di automazione e generazione che dati in input le informazioni fornite dall’utente, interagiscono con i servizi di Generative AI di AWS per la creazione dei contenuti.
        
        \item \textbf{Sprint 1 (Durata 1 settimana/40 ore)}     
        
        Attività di testing e produzione della documentazione del lavoro svolto.
    \end{itemize}
}
