%----------------------------------------------------------------------------------------
% 	USER DATA
%----------------------------------------------------------------------------------------
\newcommand{\dataApprovazione}{26 Aprile 2017}

\newcommand{\nomeStudente}{}
\newcommand{\cognomeStudente}{}
\newcommand{\matricolaStudente}{}
\newcommand{\emailStudente}{}
\newcommand{\telStudente}{}

\newcommand{\nomeTutorAziendale}{}
\newcommand{\cognomeTutorAziendale}{}
\newcommand{\emailTutorAziendale}{}
\newcommand{\telTutorAziendale}{}
\newcommand{\ruoloTutorAziendale}{}

\newcommand{\ragioneSocAzienda}{Azienda S.p.A}
\newcommand{\indirizzoAzienda}{Via Roma 1, Roma (RM)}
\newcommand{\sitoAzienda}{http://example.com/}

\newcommand{\titoloTutorInterno}{Prof.}
\newcommand{\nomeTutorInterno}{Tullio}
\newcommand{\cognomeTutorInterno}{Vardanega}


\newcommand{\scopoStage}{
Lo scopo di questo progetto di stage è ... .\\
Lo studente avrà il compito di ... .\\
}

\newcommand{\interazioneStudenteTutor}{
Lo studente, durante tutto il periodo di stage, si troverà a lavorare all'interno dello stesso ambiente lavorativo del resto del personale aziendale. Egli potrà pertanto ... .\\
Una volta la settimana, inoltre, si confronterà per un'ora con il responsabile ... .\\
}

\newcommand{\prodottiAttesi}{
Lo studente dovrà produrre una relazione scritta che illustri i seguenti punti.
\begin{enumerate}
	\item Primo punto \\
		  Descrizione. 

	\item Secondo punto \\
		  Descrizione.

	\item Terzo punto.
		  Descrizione.
\end{enumerate}

Nel qual caso in cui lo studente, in seguito all'analisi, abbia ancora tempo a sua disposizione ... .
}

\newcommand{\contenutiFormativi}{
Durante questo progetto di stage lo studente avrà occasione di approfondire le sue conoscenze nell'ambito ... .
}

\newcommand{\attivita}{
	 48 & Prima attività \\ \hdashline 
	 \multirow{4}{0cm}\\ 
	 \textit{16} & 
	 \textit{Primo compito} \\
	 \textit{8} & 
	 \textit{Secondo compito} \\ 
	 \textit{8} & 
	 \textit{Terzo compito} \\
	 \textit{16} & 
	 \textit{Quarto compito} \\ 	 
	 \hline

	 40 & Seconda attività \\ \hdashline 
	 \multirow{3}{0cm}\\ 
	 \textit{24} & 
	 \textit{Primo compito} \\
	 \textit{8} & 
	 \textit{Secondo compito} \\
	 \textit{8} & 
	 \textit{Terzo compito} \\
	 \hline
	 
	 80 & Terza attività \\ \hdashline 
	 \multirow{1}{0cm}\\ 
	 \textit{80} & 
	 \textit{Primo compito} \\
	 \hline
	 
	 52 & Quarta attività \\ \hdashline 
	 \multirow{1}{0cm}\\ 
	 \textit{52} & 
	 \textit{Secondo compito} \\
	 \hline
	 
	 80 & Quinta attività  \\ \hdashline 
	 \multirow{1}{0cm}\\ 
	 \textit{8} & 
	 \textit{Primo compito} \\
	 \hline
}

\newcommand{\totaleOre}{300}

\newcommand{\obiettiviObbligatori}{
	 \item \underline{\textit{O01}}: primo obiettivo;
	 \item \underline{\textit{O02}}: secondo obiettivo;
	 \item \underline{\textit{O03}}: terzo obiettivo;
	 
}

\newcommand{\obiettiviDesiderabili}{
	 \item \underline{\textit{D01}}: primo obiettivo;
	 \item \underline{\textit{D02}}: secondo obiettivo;
}

\newcommand{\obiettiviFacoltativi}{
	 \item \underline{\textit{F01}}: primo obiettivo;
	 \item \underline{\textit{F02}}: secondo obiettivo;
	 \item \underline{\textit{F03}}: terzo obiettivo;
}

\newcommand{\pathDiagramma}{Immagini/gantt.png}